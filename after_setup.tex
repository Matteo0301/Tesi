\chapter{After setup}

\section{Setup}
To setup the camera it is sufficient to connect the phone to the camera wifi
and add the device to the app. The app will automatically connect to the camera
and we will be able to configure the camera to connect to the local wifi.
In case the camera has been connected using the ethernet cable, it is
sufficient to add it to the app using the QR code on the back of the camera.\\\\
After having configured the camera we can use the script created to list new hosts
in a network to find the IP addresses of the camera and the phone.\\
It can be run using this command:
\begin{lstlisting}[numbers=none]
    sudo ./find_hosts -i 192.168.200.223/24
\end{lstlisting}
where the IP address is the one of the local network (which
can be found using the \textbf{ifconfig} command).\\
An example of the output of the script is the following:
\begin{lstlisting}[numbers=none]
    ...
    192.168.200.237
\end{lstlisting}
We can confirm this is the IP address of the camera by checking on
the app. The address of the phone can be found in the same manner and in our case
it was \textbf{192.168.200.226}.\\
\section{Scanning}
We can now repeat the scan of the camera to see which ports are open, now that the
configuration has changed.\\
We can run the script using the same command as before command:
\begin{lstlisting}[numbers=none]
    sudo ./scan -futav -o $FILENAME -i 192.168.200.237
\end{lstlisting}
Strangely enough, the script did not recognize the camera operating system
and MAC address the first time it was run.\\
But the second time correctly gave the following output:
\begin{lstlisting}[numbers=none]
    ... 
    MAC Address: E0:51:D8:ED:73:85 (China Dragon Technology Limited)
    Device type: general purpose
    Running: Linux 2.6.X|3.X
    OS CPE: cpe:/o:linux:linux_kernel:2.6 cpe:/o:linux:linux_kernel:3
    OS details: Linux 2.6.32 - 3.5
    Uptime guess: 0.004 days (since Tue Aug 22 17:34:17 2023)
    Network Distance: 1 hop
    ... 
\end{lstlisting}
We can see that it runs a Linux kernel between 2.6.32 and 3.5, so it's not 
the latest version, and this may lead to some security issues.\\\\
We found that the open ports are the following:
\begin{lstlisting}[numbers=none]
    1300/tcp open h323hostcallsc
    19096/udp open|filtered unknown
    3702/udp open|filtered ws-discovery
    443/tcp open ssl/https
    49167/udp open|filtered unknown
    80/tcp open http
    8000/udp open|filtered irdmi
    843/tcp open unknown
\end{lstlisting}
The script also complained about two services being unrecognized, despite having
returned data during the first scan.\\
We can see that the ports associated to the SMB protocol are not open anymore,
but we have some new ports, such as the 80 and 443 over TCP, which are used by \textbf{HTTP}
and \textbf{HTTPS}, probably for the administration interface of the camera.\\\\
Another interesting port is the number 1300, which is used by the \textbf{H.323} protocol,
which is used to send and receive audio and video over the network \cite{h323}.\\
Other open ports are the ones associated with
\textbf{ws-discovery}, a protocol to search for web services on
a network \cite{ws-discovery}, and \textbf{irdmi}, a protocol developed
by Intel that provides a remote desktop management interface for
the device.\\
As before we also have two ports that are not recognized and are therefore flagged
as unknown.\\\\
During the test phase of those ports, we got the following output:
\begin{lstlisting}[numbers=none]
    ... 

    PORT    STATE SERVICE
    443/tcp open  https
    | ssl-cert: Subject: commonName=localhost/organizationName=Example.com/stateOrProvinceName=Washington/countryName=US
    | Issuer: commonName=localhost/organizationName=Example.com/stateOrProvinceName=Washington/countryName=US
    | Public Key type: rsa
    | Public Key bits: 2048
    | Signature Algorithm: sha256WithRSAEncryption
    | Not valid before: 2015-10-29T00:57:15
    | Not valid after:  2025-10-26T00:57:15
    | MD5:   2b31:8848:da07:3366:8eba:ecf3:9c96:0be3
    |_SHA-1: 48f0:92d1:bc58:b700:b956:2fb3:5819:efee:c4e8:8eab

    ... 

    PORT   STATE SERVICE
    80/tcp open  http
    |_http-favicon: Unknown favicon MD5: 761B3CBCC87F3F8BE5FCF40C779E98B9
    | http-methods: 
    |_  Supported Methods: GET HEAD POST
    | http-title: index
    |_Requested resource was http://192.168.200.237/index.html

    ... 
\end{lstlisting}
The only useful information we can get from this output is what
HTTP methods are allowed by the server, which are \textbf{GET}, \textbf{HEAD} and \textbf{POST}, 
and some information about the SSL certificate used by the HTTPS server.\\
\section{Packet sniffing}
We can now try to sniff the traffic between the camera and the router and between
the phone and the router.\\
We can use the script created to sniff the traffic and save it to a file, and
analyze later the pcap using Wireshark.\\
We can run the script using the following command:
\begin{lstlisting}[numbers=none]
    sudo ./sniff -1 192.168.200.226 -2 192.168.200.1 -i wlo1
\end{lstlisting}
to start the sniffing process for the camera. Unfortunately, the analysis of the
pcap file did not give any useful information, so we were not able to get the source
for the firmware updates of the camera and we could not analyze it further. 
\newpage