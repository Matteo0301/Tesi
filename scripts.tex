\chapter{Scripts}
Here we will analyze the capabilities of the scripts developed for the project.\\
All the scripts are collected in a GitHub repository\cite{scripts-github}, so everyone
can find and use them and check their usage by reading the provided
documentation.
\section{Scan}
First of all we will analyze the script for the scanning and testing of the hosts on 
the network. It's a Bash script that uses \textbf{nmap} to discover hosts on the
network, search for open ports and run test script on those that are found.\\
If we run the script with the following option:
\begin{lstlisting}[numbers=none]
    scan -h
\end{lstlisting}
we will get all the options it can accept and the output will be the following:
\begin{lstlisting}[numbers=none]
    Usage: scan [OPTIONS]
        -u              perform udp scan
        -a              perform attack with default scripts
        -t              perform tcp scan
        -f              full scans
        -d              discovery mode (only -i and -o relevant)
        -v              scan for OS version
        -o FILENAME     redirect output to FILENAME
        -i IP           scan specified IP
        -m MODE         tcp scan mode
\end{lstlisting}
The \textbf{-d} option can be used to search for hosts on the network, and
only the \textbf{-i} and \textbf{-o} options will be relevant, all the others will
be discarded.\\
The \textbf{-i} option can be used to specify the IP address of the host to scan,
or of the network in case we are using the discovery mode (in which case we have to
specify the address using the CIDR notation). The \textbf{-o} option will cause 
all the outputs of the commands to be written also to that file, and not only to standard
output. In case we don't want to retain the output we can specify the \textbf{/dev/null} file.
If no file is specified it will default to "\textbf{output}".\\\\
We can pass the \textbf{-v} option to perform an OS scan on the specified
host, retrieving OS type and version.\\
The options \textbf{-t} or \textbf{-u}, which can be combined,
will perform a TCP or UDP scan respectively. The \textbf{-f} options will cause a full
scan to be performed, instead of only searching for the 1000 most common ports,
and it will affect bot UDP and TCP scans.
We can also specify the TCP scan mode to use with the \textbf{-m} option, in which
case it's not necessary to pass \textbf{-t}, because it is subsumed.\\
If the \textbf{-a} option is specified, the script will run the default nmap scripts
on the ports that are found open.\\\\
The script will also print every command executed and it's output, and it will
print what ports are found open before starting to, eventually, test them.
\section{New hosts discovery}
This script uses the \textbf{nmap} tool as before, to discover the newly connected
hosts on the network.\\
To obtain a help message from the script we can use the same syntax as before:
\begin{lstlisting}[numbers=none]
    find_hosts -h
\end{lstlisting}
and we will get the following output:
\begin{lstlisting}[numbers=none]
    Usage: find_hosts [OPTIONS]
    -h              print this help message
    -f FILE         get old IP addresses from FILE
    -i IP           scan specified IP
\end{lstlisting}
We see that we can use the \textbf{-i} option (required) to specify which network to scan,
passing an IP address in CIDR notation. The script will perform a first scan and
print the hosts that are found. It will then wait for the user to press enter,
after the new devices have been connected to the network, and it will perform
another scan, printing the IP addresses of the new hosts.\\\\
The process of waiting for the user to press enter can be avoided by passing
the \textbf{-f} option, which will cause the script to read the initial IP addresses
from a file (one address per line) and proceed directly to the second scan.
\section{Sniffing}
This script relies on the \textbf{ettercap}\cite{ettercap-home}\cite{ettercap-kali}\cite{ettercap-man} tool to sniff the packets between the two specified
hosts. It doesn't add much functionality, but it's useful to automate
and standardize the process, also simplifying its usage.\\
If we run the script with the following option:
\begin{lstlisting}[numbers=none]
    sniff -h
\end{lstlisting}
we will see the following output:
\begin{lstlisting}[numbers=none]
    Usage: sniff [OPTIONS]
    -h              print this help message
    -1 IP           specify first IP address
    -2 IP           specify second IP address
    -o FILE         write output to FILE
    -i IFACE        which interface to listen on
    -m MODE         what mode to use for mitm (default is arp)
\end{lstlisting}
We can use the \textbf{-1} and \textbf{-2} options to specify the IP addresses of the
two hosts we want to sniff (both are mandatory). The \textbf{-o} option can be used
to specify the file to which we want to write the output, and if it's not specified
it will default to "\textbf{output.pcap}".\\
The \textbf{-i} option can be used to specify the interface to listen on, and if it's
not specified it will default to "\textbf{wlo1}".\\
Finally, the \textbf{-m} option can be used to specify the mode to use for the mitm
attack, defaulting to \textbf{arp} if not specified.\\
This script, once started, will open the textual user interface of ettercap,
and will wait for the user to press \textbf{q} to quit, saving all packets
to the target output file.\\
\newpage