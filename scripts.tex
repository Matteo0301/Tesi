\chapter{Scripts}
Here we will analyze the capabilities of the scripts developed for the project.\\\\
First of all we will analyze the script for the scanning and testing of the hosts on 
the network. It's a Bash script that uses \textbf{nmap} to discover hosts on the
network, search for open ports and run test script on those that are found.\\
If we run the script with the following syntax:
\begin{lstlisting}[numbers=none]
    scan -h
\end{lstlisting}
we will get all the options it can accept and the output will be the following:
\begin{lstlisting}[numbers=none]
    Usage: scan [OPTIONS]
        -u              perform udp scan
        -a              perform attack with default scripts
        -t              perform tcp scan
        -f              full scans
        -d              discovery mode (only -i and -o relevant)
        -v              scan for OS version
        -o FILENAME     redirect output to FILENAME
        -i IP           scan specified IP
        -m MODE         tcp scan mode
\end{lstlisting}
The \textbf{-d} option can be used to search for hosts on the network, and
only the \textbf{-i} and \textbf{-o} options will be relevant, all the others will
be discarded.\\
The \textbf{-i} option can be used to specify the IP address of the host to scan,
or of the network in case we are using the discovery mode (in which case we have to
specify the address using the CIDR notation). The \textbf{-o} option will cause 
all the outputs of the commands to be written also to that file, and not only to standard
output. In case we don't want to retain the output we can specify the \textbf{/dev/null} file.
If no file is specified it will default to "\textbf{output}".\\
We can pass the \textbf{-v} option to perform an OS scan on the specified
host, retrieving OS type and version.\\
The script can be passed the \textbf{-t} or \textbf{-u} options, also in combination,
to perform a TCP or UDP scan respectively. The \textbf{-f} options will cause a full
scan to be performed, instead of only searching for the 1000 most common ports.
We can also specify the TCP scan mode to use with the \textbf{-m} option, in which
case it's not necessary to pass \textbf{-t}, because it is subsumed.\\
If the \textbf{-a} option is specified, the script will run the default nmap scripts
on the ports that are found open.\\
\newpage