\chapter{Before device setup}
To bring up the device it's sufficient to connect it to the power supply, then it
will start emitting periodically a "beep" sound to indicate that it's ready to be configured.\\
Then we need to connect our phone and our computer to the device's wifi network:
the SSID is "IPC\_***" and the default password is "12345678".\\\\
After that we can start to scan the device using the script we wrote. First of all
we need to get the address IP of the device, and we can do that using the command:
\begin{lstlisting}[numbers=none]
    ip route
\end{lstlisting}
The output will be something like this:
\begin{lstlisting}[numbers=none]
    default via 192.168.66.1 dev wlo1 proto dhcp src 192.168.66.100 metric 20600 
    ...
\end{lstlisting}
So we can conclude that the IP address of the device is \textbf{192.168.66.100}.\\\\
Now we can start the script using the command:
\begin{lstlisting}[numbers=none]
    sudo ./scan -futav -o $FILENAME -i $IP
\end{lstlisting}
where \textbf{\$FILENAME} is the name of the file where we want to save the output
and \textbf{\$IP} is the IP address of the device. As previously
said this command requires to be run using root privileges
because of the use of the nmap command.\\\\
Regarding the OS version we got the following output:
\begin{lstlisting}[numbers=none]
    No exact OS matches for host (If you know what OS is running on it, see https://nmap.org/submit/ ).
\end{lstlisting}
from which we can understand that nmap was not able to identify the OS running
on the device.\\\\
We can also see that the device exposes various open ports:
\begin{lstlisting}[numbers=none]
    137/udp open netbios-ns
    138/udp open|filtered netbios-dgm
    139/tcp open netbios-ssn
    1716/tcp open xmsg
    1716/udp open|filtered xmsg
    41058/udp open|filtered unknown
    445/tcp open microsoft-ds
    445/tcp open netbios-ssn
    51225/udp open|filtered unknown
    5353/udp open zeroconf
    6566/tcp open sane-port
    6566/tcp open tcpwrapped
    8828/tcp open unknown
\end{lstlisting}
We can see that the UDP ports number 137 and 138 and the 139 TCP port are open,
which means that the device is exposing the \textbf{netbios} protocol, and the ports are used
respectively to provide name lookup (137), the datagram service (138) and the session
service (139) \cite{netbios-ws}\cite{netbios-smb}.\\
We can also see that the port 1716 is open for both TCP and UDP (although filtered), and exposes
\textbf{xmsg} services, used to share information via XML documents.\\\\
Then we have port 445 (both UDP and TCP) which is associated with \textbf{SMB}:
a protocol developed by Microsoft to share files between machines over network.\\
The port for the \textbf{zeroconf} service is used to dynamically configure the
hosts on the network \cite{zeroconf}, and \textbf{sane-port} is a protocol used
to share scanner devices with other hosts. The latter is also reported as a tcpwrapped port, which means
that the port is protected by TCP Wrappers, an ACL system used to filter incoming packages
using rules (such as allow only some hosts) in a similar way to firewalls. \\\\
Then we also have two ports that are unknown, because they are not standard, therefore we don't know what services
are associated with them.\\\\
Then we have the output associated with the scripts run by nmap, which reports the following:
\begin{lstlisting}[numbers=none]
    ...
    PORT    STATE SERVICE
    139/tcp open  netbios-ssn

    Host script results:
    |_nbstat: NetBIOS name: , NetBIOS user: <unknown>, NetBIOS MAC: <unknown> (unknown)
    | smb2-security-mode: 
    |   3:1:1: 
    |_    Message signing enabled but not required
    | smb2-time: 
    |   date: 2023-08-14T14:36:09
    |_  start_date: N/A

    ...
    PORT    STATE SERVICE
    445/tcp open  microsoft-ds

    Host script results:
    | smb2-time: 
    |   date: 2023-08-14T14:36:51
    |_  start_date: N/A
    | smb2-security-mode: 
    |   3:1:1: 
    |_    Message signing enabled but not required
    |_clock-skew: -1s
    |_nbstat: NetBIOS name: , NetBIOS user: <unknown>, NetBIOS MAC: <unknown> (unknown)

    ...
\end{lstlisting}
This output is generated by the \textbf{smb2-security-mode} script \cite{smb-script},
which complains about the fact that the message signing is enabled but not required,
and this could lead to security issues.\\\\
Then we have the output of the \textbf{nbstat} script \cite{nbstat}, which
unfortunately doesn't find any hostname and is not able to retrieve the netbios user or the MAC address.
\newpage