\chapter{Wi-Fi access}
In the thesis on which \cite{previouswork} this work is based it was suggested
to use the Fern wifi cracker \cite{fern} to gain access to the target newtwork.
Tis tool is basically a GUI for the aircrack-ng suite \cite{aircrack} and
exposes all the features of the command line tool. It can in fact perform
a lot of other attacks, aside from the WPA/WPA2 cracking, such as:
\begin{itemize}
    \item Automatic attacks on the access point
    \item MITM attacks
    \item Bruteforce attacks using protocols such as FTP, HTTP/HTTPS and TELNET
    \item Deauthentication, access point spoofing and replay attacks
\end{itemize}
Nevertheless, as the goal of this work is to automate the process, the command
line tool may prove to be more useful, as it can be easily integrated in a
script.\\
The Aircrack-ng suite is the most portable of the tools presented here, as it
can be used from virtually every operating system broadly used today, such as 
Windows, MacOS, a lot of Linux distributions and even from BSD operating systems.\\
There is also a Docker container for running it without installing it on the
main system. It can be installed using the right packaged version for the host OS
or by compiling it from source \cite{aircrack-git}.\\\\
Other tools that can be employed instead of aircrack-ng are:
\begin{itemize}
    \item Reaver \cite{reaver}: a tool that exploits a flaw in the WPS
        implementation of some routers using the WPS Pixie-Dust and other brute-force attacks
    \item Wifite \cite{wifite}: a python script that is a frontend for a lot of 
        tools, including aircrack-ng, reaver, cowpatty, pyrit, tshark, etc. It
        is therefore perfect to automate the process of vulnerability assessment
        of a wifi network
    \item Bully \cite{bully}: similar to Reaver
\end{itemize}
The suggested tool is Wifite, as it is the most complete and it is a frontend
to all other tools presented here. It is also a command line program, so it
can be easily used in a script and it is also very easy to use because of the
almost complete automation of the process. Unfortunately it needs many dependencies
to reach its full potential, but many of them are optional.
It can be downloaded using git in the following way:
\begin{lstlisting}[numbers=none]
    git clone https://github.com/derv82/wifite2.git
\end{lstlisting}
It can then be run opening the terminal in the cloned directory and running:
\begin{lstlisting}[numbers=none]
    sudo ./Wifite.py
\end{lstlisting}
\newpage