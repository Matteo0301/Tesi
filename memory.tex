\chapter{Memory analysis}
\section{Dump creation}
In this chapter we will focus on memory analysis: a type of vulnerability assessment not part of the framework on which
this thesis is based. The goal will be to retrieve the memory dump of a running device
and analyze it to find out if it's vulnerable.\\\\
We will assume the target device runs a Linux based operating system, because
in case of IoT devices this is the most common scenario. If the device has a
proprietary OS the analysis will be more difficult, but the same steps will apply
(dump creation, retrieval and analysis).\\
We also assume the attacker already has a working root shell on the device,
as all operations of memory read and kernel module loading are only allowed in
case of privileged access.\\\\
In order to do this we first have to extract the memory dump from the device
and put it in a file. This can be done using the dd command, present on all
Linux installations.\\ It can be used for example to copy the content of \textbf{/dev/mem}
calling it with the following syntax:
\begin{lstlisting}[numbers=none]
    dd if=/dev/mem of=memdump bs=1M count=1
\end{lstlisting}
This command will copy the specified number of blocks (count) of the specified
size (bs) to the specified output file.\\\\
Unfortunately we need a suitable device from which to extract the memory and \textbf{/dev/mem}
may have been configured to allow only a small amount of memory to be read:
this depends on the OS configuration so it's impossible to know in advance.\\
This limit may be of 1Mb or 1Gb, depending on the architecture and configuration, but it's only
available since version 2.6.26 of the Linux kernel\cite{dev-mem-man} and only
if the kernel has been compiled with the \textbf{CONFIG\_STRICT\_DEVMEM} option.\\
On the x86 architecture it almost completely disallows memory access, allowing
only some memory addresses, such as memory mapped PCI devices, to be read.
The fact that the restriction is only available since a specific version may be an advantage for an attacker
because it means that older versions of the kernel (usually used on IoT devices) may not have it.\\\\
A similar device to \textbf{/dev/mem} is \textbf{/dev/kmem}, which works the same 
way, but uses kernel virtual memory addresses instead of physical memory addresses.\\
Moreover it's only available if the kernel has been compiled with the \textbf{CONFIG\_DEVKMEM} option, so 
it's not always available and not a good choice for our purposes.\\\\
Another device from which we can obtain a memory dump is \textbf{/dev/fmem}\cite{dev-fmem}, which
can be created with the use of a kernel module. It allows to read the whole physical memory
without restrictions, but needs the installation of the kernel module and is not guaranteed to 
work with the specific kernel version or architecture. In fact it's not been updated
for a while, but this may not be a problem, because IoT devices rarely ship with the latest
kernel versions.\\
The last option that we'll analyze is the use of the \textbf{LiME} (Linux Memory Extractor)\cite{lime}.
It's the most advanced option, because it offers a lot of convenient
features that can help to retrieve the memory dump, with the downside (as before)
of needing the installation of a kernel module.\\
It can be used with the following syntax:
\begin{lstlisting}[numbers=none]
    insmod ./lime.ko "path=<outfile | tcp:<port>> format=<raw|padded|lime> [digest=<digest>] [dio=<0|1>]"
\end{lstlisting}
The module is able to:
\begin{itemize}
    \item Dump memory to a file or listen on a tcp port and send the memory dump to a remote host
    \item Use different formats for the memory dump
    \item Use different digest algorithms to verify the integrity of the memory dump 
\end{itemize}
\subsubsection{QEMU memory dump}
In case the attacker succeeded in emulating the device image with QUEMU (for example
using \textbf{FYRMADYNE}\cite{firmadyne}, as suggested in \cite{previouswork}), then it is possible
to obtain a memory dump of the emulated device.\\
This can be done with the \textbf{pmemsave} or the \textbf{dump-guest-memory} quemu commands.\\
\subsubsection{File retrieve}
Once we have created the memory dump we need to retrieve it from the device.\\
If LiMe was not used than it's possible to use the commands \textbf{netcat}\cite{netcat} or \textbf{scp}\cite{scp}
to upload the file to the attacker's machine.\\
As we assume the target device to be a Linux machine, scp will probably be installed
by default, while netcat will need to be installed.
\section{Analysis}
To analyze the dump file it's possible to use the \textbf{Volatility}\cite{volatility} framework, which is a collection
of python tools to analyze memory images and extract information from them.
We suggest using the original version, but there is also another one, more up
to date and written in Python 3 instead of Python 2, called \textbf{Volatility3}\cite{volatility3},
still in development.\\\\
It can be used for example to see open files and running processes at the time of the dump, but
has many capabilities, listed on the official documentation. Volatility supports
different file formats (raw, LiME, etc.) and different operating systems (Windows, Linux, etc.),
but for systems other than Windows you need to configure the right profile,
or even to write your own (in case there is no suitable one).\\\\
For Linux there are already some profiles available, but it's very likely that
an attacker will need to create his own ones, because of the huge variety of 
IoT devices. However the instructions can be found on the official documentation
and on other good tutorials\cite{volatility-conf}\cite{volatility-tutorial}.\\\\
Once the profile is configured it's possible to use the framework to analyze the dump file, to
retrieve all important information about the system. It's also possible to 
automate the process using the tool inside scripts, because it's a fully automated
framework.\\\\
To analyze the file we can use the following command:
\begin{lstlisting}[numbers=none]
    python2 vol.py -f <dump_file> <plugin_name> --profile=<profile_name>
\end{lstlisting}
The plugin name specifies which action to perform on the file, so we can
search for different information changing this parameter. A list of all Linux plugins can be
obtained with this command:
\begin{lstlisting}[numbers=none]
    python2 vol.py --info  | grep linux_
\end{lstlisting}
\newpage